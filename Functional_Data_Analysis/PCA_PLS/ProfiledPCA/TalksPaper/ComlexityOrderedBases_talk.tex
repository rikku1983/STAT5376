\documentclass[11pt]{beamer}

\usetheme{Darmstadt}

\usepackage{times}
\usefonttheme{structurebold}

\usepackage[english]{babel}
\usepackage{pgf,pgfarrows,pgfnodes,pgfautomata,pgfheaps}
\usepackage{amsmath,amssymb}
\usepackage[latin1]{inputenc}
\usepackage{multicol}

\input{slabbrev}

\title{Data Smoothing with Complexity-ordered Basis Functions}

\author{Jim Ramsay, McGill University}
\date{}

\begin{document}

\begin{frame}

\maketitle

\end{frame}

%  --------------------------------------------------------------------

\begin{frame}

\frametitle{The essential idea}

\bi
  \item An attractive aspect of principal components analysis of functional data is that each successive
  component is more functionally complex than its predecessor.
  \item Principal components are ordered in terms of complexity.
  \item We can achieve the same thing with fixed non-empirical basis systems.
\ei

\end{frame}

%  --------------------------------------------------------------------

\begin{frame}

\frametitle{How we choose basis systems}

\bi
  \item We choose basis function systems $\phi_1, \ldots, \phi_K$ for all sorts of reasons:
  \bi
    \item to match characteristics of our data, e. g. fourier bases for periodic data
    \item fast computation for large records, e. g. b-splines
    \item lowest possible dimensionality, e. g. principal component bases
    \item substantive interpretability
  \ei
  \item Unlike PCA, these basis functions are not typically ordered in terms of complexity.
  \item Unlike PCA, we can't just use the first basis function, the first two, and etc. until we find that we like the fit.
  \item We need to work with them all simultaneously.
\ei

\end{frame}

%  --------------------------------------------------------------------

\begin{frame}

\frametitle{The typical roughness-penalized smoothing problem}

\bi
  \item The function $x$ fitting the data $y_1, \ldots, y_n$ is defined by the basis function expansion
  \[
    x(t) = \sum_k^K c_k \phi_k(t) = \cbold'\phibold(t)
  \]
  \item We often use roughness penalties to enforce smoothness, e. g.
  \[
    \min_x \{ \sum_i[y_i - x(t_i)]^2 + \lambda \int ](D^2 x)(t)]^2 \, dt \}
  \]
  \item Here, ``roughness'' is defined as the total squared curvature of $s$.
\ei

\end{frame}

%  --------------------------------------------------------------------

\begin{frame}

\frametitle{The structure of roughness penalty matrix $\Rbold$}

\bi
  \item The second roughness penalty term can be re-expressed as
  \[
    P(x) =  \lambda \cbold' \Rbold \cbold
  \
  \item  Order $K$ roughness penalty matrix $\Rbold$ contains, when roughness is defined as the size of the
   second derivative, $D^2 x$, the basis curvature inner products
  \[
    \int D^2\phi_k(t) D^2\phi_\ell(t) \, dt
  \]
  \item The internal structure of $\Rbold$ contains information about the complexity of the
  function $x$ defined by coefficient vector $\cbold$.
\ei

\end{frame}

%  --------------------------------------------------------------------

\begin{frame}

\frametitle{The kernel of  $\Rbold$}

\bi
  \item The solutions of the equation
  \[
    \Rbold \cbold = 0
  \]
  correspond to functions that are ``hypersmooth'' in the sense of having zero roughness penalty.
  \item We say that $\cbold$ is in the \emph{kernel} of linear transformation $\Rbold$
  \item Defining roughness as the size of the curvature $D^2 x(t)$ means that this equation defines
  $x$ as a straight line having zero curvature.
  \item Different measures of roughness correspond to different ``kernel functions''
  \item Harmonic curvature, $D^3 x + \omega^2 x$, has as kernel functions 
  \[
    x(t) = 1, \sin(\omega t), \ \mbox{and} \ \cos(\omega t)
  \]
  \item Ideally, our choice of ``rougness'' should imply kernel functions that come close to the shape of our data.
\ei

\end{frame}

%  --------------------------------------------------------------------

\begin{frame}

\frametitle{The eigenvalues and eigenvectors of  $\Rbold$}

\bi
  \item We can generalize the kernel equation to
  \[
    \Rbold \cbold = \mu \cbold
  \]
  \item A solution $\cbold_\ell$ this equation is an eigenvector, and the corresponding $\mu_\ell$ is its eigenvalue.
  \item $\cbold_\ell$ defines a function with a particular type of roughness, and $\mu_\ell$ indicates the size
  of the penalty placed on that kind of roughness.  
  \item Kernel functions are ``hypersmooth'' because $\mu = 0$ for that kind of function shape.
\ei

\end{frame}

%  --------------------------------------------------------------------

\begin{frame}

\frametitle{Ordering eigenfunction by complexity}

\bi
  \item Contrary to the usual practice, let's order the eigenvalues in \emph{increasing} order.  
  \item Because $\Rbold$ contains inner products, $\Rbold$ is symmetric and these eigenvalues will be non-negative.
  \item The first few will have eigenvalues 0.  The corresponding functions show us what the kernel functions are, 
  something that isn't that obvious for choices of linear differential operators more exotic than $D^2$.
  \item As we progress through the increasing eigenvalues, the corresponding eigenfunctions $\cbold_\ell' \phibold$ will 
  more and more rough.
  \item $\cbold_K' \phibold$ will be the worst of the lot.
\ei

\end{frame}

%  --------------------------------------------------------------------

\begin{frame}

\frametitle{The Fourier series case}

\bi
  \item For those that like their examples mathematical, consider the roughness definition $D^3 x + \omega^2 \D x$
  \item We already pointed out that the first three are $1, \sin(\omega t), \ \mbox{and} \ \cos(\omega t)$, corresponding
  to $\mu_\ell = 0.$
  \item The 4th is $\sin(2 \omega t)$, oscillating twice as often as the 2nd.
  \item The $2m + 1$th is $\cos m \omega t$, oscillating $m$ times more often that the 3rd.
\ei

\end{frame}

%  --------------------------------------------------------------------

\begin{frame}

\frametitle{The B-spline case}

\bi
  \item The simple differential operator $D^m$ implies a spline function as $x$, which is piece-wise a polynomial of
  order $2m$ (of degree $2m-1$).  The polynomial segments are joined end-to-end at locations called \emph{knots.}
  \item Consider an order 4 B-spline basis over the interval $[0,10]$ with knots at 2, 4, 6, and 8.  There are 8 basis functions for this system, and they are shown in the following figure.
\ei

\end{frame}

%  --------------------------------------------------------------------

\begin{frame}

\include





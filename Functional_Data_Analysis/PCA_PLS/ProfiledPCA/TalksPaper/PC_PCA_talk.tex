\documentclass[11pt]{beamer}

\usetheme{Darmstadt}

\usepackage{times}
\usefonttheme{structurebold}

\usepackage[english]{babel}
%\usepackage{pgf,pgfarrows,pgfnodes,pgfautomata,pgfheaps}
\usepackage{amsmath,amssymb}
\usepackage[latin1]{inputenc}
\usepackage{multicol}

%
%  This version the file slabbrev.tex is designed to work with
%  the beamer package
%

\newcommand{\ns}{
  \end{slide}
  \begin{slide}
  \sffamily
}
\newcommand{\bs}{
  \begin{slide}
  \sffamily
}
\newcommand{\es}{
  \end{slide}
}

\newcommand{\bi}{
  \begin{itemize}
}

\newcommand{\ei}{
  \end{itemize}
}

%
%  Inner product macro
%
\newcommand{\IP}[2]{ \langle #1 , #2 \rangle }
%
\newcommand{\twobytwo}[4]{\left ( \begin{array}{cc}#1&#2\\#3&#4 \end{array}
\right ) }
\newcommand{\twobyone}[2]{\left ( \begin{array}{c}#1 \\ #2 \end{array}
\right ) }
\newcommand{\twobytwob}[4]{\left [ \! \begin{array}{cc}#1&#2\\#3&#4 \end{array}
\! \right ] }
\newcommand{\twobyoneb}[2]{\left [\! \begin{array}{c}#1 \\ #2 \end{array}
\! \right ] }
\renewcommand{\ne}{\! \not \, = }
\newcommand{\be}[1]{\begin{equation}\label{#1}}
\newcommand{\ee}{\end{equation}}
\newcommand{\cJ}{{\cal J}}
\newcommand{\cK}{{\cal K}}
\newcommand{\cM}{{\cal M}}
\newcommand{\cS}{{\cal S}}
\newcommand{\cT}{{\cal T}}
\newcommand{\cV}{{\cal V}}
\newcommand{\cZ}{{\cal Z}}
\newcommand{\muh}{\hat{\mu}}
\newcommand{\Gamh}{\hat{\Gamma}}
\newcommand{\gamh}{\hat{\gamma}}
\newcommand{\gamt}{\tilde{\gamma}}
\newcommand{\xit}{\tilde{\xi}}
\newcommand{\ut}{\tilde{u}}
\newcommand{\vt}{\tilde{v}}
\newcommand{\lamh}{\hat{\lambda}}
\newcommand{\gamb}{\mbox{\boldmath$\gamma$}}
\newcommand{\Gamb}{\mbox{\boldmath$\Gamma$}}
\newcommand{\betb}{\mbox{\boldmath$\beta$}}
\newcommand{\bX}{{\bf X}}
\newcommand{\bg}{{\bf g}}

\newcommand{\Ddo}   {\mbox{D}}
\newcommand{\Ldo}   {\mbox{L}}

\newcommand{\ROUPEN}   {\mbox{\tt ROUPEN}}
\newcommand{\PEN}      {\mbox{\tt PEN}}
\newcommand{\PENRSS}   {\mbox{\tt PENRSS}}
\newcommand{\REGSSE}   {\mbox{\tt REGSSE}}
\newcommand{\PCASSE}   {\mbox{\tt PCASSE}}
\newcommand{\PCAPSV}   {\mbox{\tt PCAPSV}}
\newcommand{\LMSSE}    {\mbox{\tt LMSSE}}
\newcommand{\PENSSE}   {\mbox{\tt PENSSE}}
\newcommand{\SMPENSSE} {\mbox{\tt SMPENSSE}}
\newcommand{\RSQ}      {\mbox{\tt RSQ}}
\newcommand{\FRATIO}   {\mbox{\tt FRATIO}}
\newcommand{\SMSSE}    {\mbox{\tt SMSSE}}
\newcommand{\LOCSSE}   {\mbox{\tt SMLOCSSE}}
\newcommand{\Kern}     {\mbox{\tt Kern}}
\newcommand{\Triang}   {\mbox{\tt Triang}}
\newcommand{\cov}      {\mbox{\tt cov}}
\newcommand{\corr}     {\mbox{\tt corr}}
\newcommand{\var}      {\mbox{\tt var}}
\newcommand{\GCV}      {\mbox{\tt GCV}}
\newcommand{\CV}       {\mbox{\tt CV}}
\newcommand{\DF}       {\mbox{\tt DF}}
\newcommand{\GFR}      {{\tt GFR}}
\newcommand{\GOP}      {{\tt GOP}}
\newcommand{\KUC}      {{\tt KUC}}
\newcommand{\LMISE}    {\mbox{\tt LMISE}}
\newcommand{\sign}     {\mbox{\tt sign}}
\newcommand{\MINEIG}   {\mbox{\tt MINEIG}}
\newcommand{\Height}   {\mbox{\tt Height}}
\newcommand{\Acceleration}   {\mbox{\tt Acceleration}}
\newcommand{\Temp}     {\mbox{\tt Temp}}
\newcommand{\Prec}     {\mbox{\tt Prec}}
\newcommand{\Precmn}   {\mbox{\tt Precmn}}
\newcommand{\Precstar} {\mbox{{\tt Prec}^{*}}}
\newcommand{\LPrec}    {\mbox{\tt LPrec}}
\newcommand{\Hi}       {\mbox{\footnotesize \tt H}}
\newcommand{\Kn}       {\mbox{\footnotesize \tt K}}
\newcommand{\Hipsmall}  {\mbox{\footnotesize \tt Hip}}
\newcommand{\Kneesmall} {\mbox{\footnotesize \tt Knee}}
\newcommand{\Hip}      {\mbox{\tt Hip}}
\newcommand{\Knee}     {\mbox{\tt Knee}}
\newcommand{\Hipmn}    {\mbox{\tt Hipmn}}
\newcommand{\Kneemn}   {\mbox{\tt Kneemn}}
\newcommand{\Angles}   {\mbox{\tt Angles}}
\newcommand{\ScriptX}  {\mbox{\tt ScriptX}}
\newcommand{\ScriptY}  {\mbox{\tt ScriptY}}
\newcommand{\GDP}      {\mbox{\tt GDP}}
\newcommand{\Bias}     {\mbox{\tt Bias}}
\newcommand{\IMSE}     {\mbox{\tt IMSE}}
\newcommand{\RMSE}     {\mbox{\tt RMSE}}
\newcommand{\Var}      {\mbox{\tt Var}}
\newcommand{\Stt}      {\mbox{\tt S}}
\newcommand{\Ctt}      {\mbox{\tt C}}
\newcommand{\SSE}      {\mbox{\tt SSE}}
\newcommand{\SSY}      {\mbox{\tt SSY}}
\newcommand{\SAD}      {\mbox{\tt SAD}}
\newcommand{\MSE}   {\mbox{\tt MSE}}
\newcommand{\PSSE}  {\mbox{\tt PSSE}}
\newcommand{\MSR}   {\mbox{\tt MSR}}
\newcommand{\tPDA}  {\mbox{\tiny PDA}}
\newcommand{\tPCA}  {\mbox{\tiny PCA}}
\newcommand{\Force} {\mbox{\tt Force}}
\newcommand{\ForceX}    {\mbox{\tt ForceX}}
\newcommand{\ForceY}    {\mbox{\tt ForceY}}
\newcommand{\Contrast}  {\mbox{\tt Contrast}}

\newcommand{\rhat}     {\hat{r}}
\newcommand{\uhat}     {\hat{u}}
\newcommand{\vhat}     {\hat{v}}
\newcommand{\xhat}     {\hat{x}}
\newcommand{\yhat}     {\hat{y}}
\newcommand{\Fhat}     {\hat{F}}
\newcommand{\Mhat}     {\hat{M}}
\newcommand{\alphahat} {\hat{\alpha}}
\newcommand{\betahat}  {\hat{\beta}}

\newcommand{\xtilde}   {\tilde{x}}
\newcommand{\etilde}   {\tilde{e}}
\newcommand{\ktilde}   {\tilde{k}}

\newcommand{\lcal}     {\ell}
\newcommand{\Bcal}     {\mbox{$\cal B$}}
\newcommand{\Fcal}     {\mbox{$\cal F$}}
\newcommand{\Gcal}     {\mbox{$\cal G$}}
\newcommand{\Hcal}     {\mbox{$\cal H$}}
\newcommand{\Lcal}     {\mbox{$\cal L$}}
\newcommand{\Pcal}     {\mbox{$\cal P$}}
\newcommand{\Rcal}     {\mbox{$\cal R$}}
\newcommand{\Tcal}     {\mbox{$\cal T$}}
\newcommand{\Ucal}     {\mbox{$\cal U$}}
\newcommand{\Ycal}     {\mbox{$\cal Y$}}

\newcommand{\TcalX}     {{{\cal T}_{\!X}}}
\newcommand{\TcalY}     {{{\cal T}_{\!Y}}}


\newcommand{\spann}    {\mbox{span}}
\newcommand{\diag}     {\mbox{diag}}
\newcommand{\Exp}      {\mbox{E}}

\newcommand{\abold}    {\mbox{\bf a}}
\newcommand{\bbold}    {\mbox{\bf b}}
\newcommand{\cbold}    {\mbox{\bf c}}
\newcommand{\dbold}    {\mbox{\bf d}}
\newcommand{\ebold}    {\mbox{\bf e}}
\newcommand{\fbold}    {\mbox{\bf f}}
\newcommand{\gbold}    {\mbox{\bf g}}
\newcommand{\hbold}    {\mbox{\bf h}}
\newcommand{\kbold}    {\mbox{\bf k}}
\newcommand{\rbold}    {\mbox{\bf r}}
\newcommand{\sbold}    {\mbox{\bf s}}
\newcommand{\tbold}    {\mbox{\bf t}}
\newcommand{\ubold}    {\mbox{\bf u}}
\newcommand{\vbold}    {\mbox{\bf v}}
\newcommand{\wbold}    {\mbox{\bf w}}
\newcommand{\xbold}    {\mbox{\bf x}}
\newcommand{\ybold}    {\mbox{\bf y}}
\newcommand{\zbold}    {\mbox{\bf z}}
\newcommand{\Abold}    {\mbox{\bf A}}
\newcommand{\Bbold}    {\mbox{\bf B}}
\newcommand{\Cbold}    {\mbox{\bf C}}
\newcommand{\Dbold}    {\mbox{\bf D}}
\newcommand{\Ebold}    {\mbox{\bf E}}
\newcommand{\Fbold}    {\mbox{\bf F}}
\newcommand{\Gbold}    {\mbox{\bf G}}
\newcommand{\Hbold}    {\mbox{\bf H}}
\newcommand{\Ibold}    {\mbox{\bf I}}
\newcommand{\Jbold}    {\mbox{\bf J}}
\newcommand{\Kbold}    {\mbox{\bf K}}
\newcommand{\Lbold}    {\mbox{\bf L}}
\newcommand{\Mbold}    {\mbox{\bf M}}
\newcommand{\Nbold}    {\mbox{\bf N}}
\newcommand{\Pbold}    {\mbox{\bf P}}
\newcommand{\Qbold}    {\mbox{\bf Q}}
\newcommand{\Rbold}    {\mbox{\bf R}}
\newcommand{\Sbold}    {\mbox{\bf S}}
\newcommand{\Tbold}    {\mbox{\bf T}}
\newcommand{\Ubold}    {\mbox{\bf U}}
\newcommand{\Vbold}    {\mbox{\bf V}}
\newcommand{\Wbold}    {\mbox{\bf W}}
\newcommand{\Xbold}    {\mbox{\bf X}}
\newcommand{\Ybold}    {\mbox{\bf Y}}
\newcommand{\Zbold}    {\mbox{\bf Z}}


\newcommand{\onebold}   {\mbox{\bf 1}}
\newcommand{\zerobold}  {\mbox{\bf 0}}

\newcommand{\xhatbold} {\hat{\mbox{\bf x}}}
\newcommand{\yhatbold} {\hat{\mbox{\bf y}}}
\newcommand{\Ahatbold} {\hat{\mbox{\bf A}}}
\newcommand{\Bhatbold} {\hat{\mbox{\bf B}}}
\newcommand{\Qhatbold} {\hat{\mbox{\bf Q}}}
\newcommand{\Rhatbold} {\hat{\mbox{\bf R}}}
\newcommand{\Yhatbold} {\hat{\mbox{\bf Y}}}

\newcommand{\cboldhat}    {\hat{\mbox{\bf c}}}

\newcommand{\alphabold}   {\mbox{\boldmath${\alpha}$}}
\newcommand{\betabold}    {\mbox{\boldmath${\beta}$}}
\newcommand{\deltabold}   {\mbox{\boldmath${\delta}$}}
\newcommand{\Deltabold}   {\mbox{\boldmath${\Delta}$}}
\newcommand{\epsilonbold} {\mbox{\boldmath${\epsilon}$}}
\newcommand{\gammabold}   {\mbox{\boldmath${\gamma}$}}
\newcommand{\lambdabold}  {\mbox{\boldmath${\lambda}$}}
\newcommand{\Lambdabold}  {\mbox{\boldmath${\Lambda}$}}
\newcommand{\mubold}      {\mbox{\boldmath${\mu}$}}
\newcommand{\thetabold}   {\mbox{\boldmath${\theta}$}}
\newcommand{\phibold}     {\mbox{\boldmath${\phi}$}}
\newcommand{\Phibold}     {\mbox{\boldmath${\Phi}$}}
\newcommand{\psibold}     {\mbox{\boldmath${\psi}$}}
\newcommand{\Psibold}     {\mbox{\boldmath${\Psi}$}}
\newcommand{\Sigb}        {\mbox{\boldmath${\Sigma}$}}
\newcommand{\xibold}      {\mbox{\boldmath${\xi}$}}

\newcommand{\thetaboldhat} {\hat{\mbox{\boldmath${\theta}$}}}

\newcommand{\xbar}     {\bar{x}}
\newcommand{\ybar}     {\bar{y}}
\newcommand{\Pbar}     {\bar{P}}

\newcommand{\gtilde}       {\tilde{g}}
\newcommand{\ntilde}       {\tilde{n}}
\newcommand{\alphatilde}   {\tilde{\alpha}}
\newcommand{\betatilde}    {\tilde{\beta}}

\newcommand{\estar}    {e^{*}}
\newcommand{\fstar}    {f^{*}}
\newcommand{\ustar}    {u^{*}}
\newcommand{\vstar}    {v^{*}}
\newcommand{\xstar}    {x^{*}}
\newcommand{\ystar}    {y^{*}}
\newcommand{\Astar}    {A^{*}}
\newcommand{\Bstar}    {B^{*}}
\newcommand{\Estar}    {E^{*}}
\newcommand{\Fstar}    {F^{*}}
\newcommand{\Lstar}    {L^{*}}
\newcommand{\Pstar}    {P^{*}}
\newcommand{\Ustar}    {U^{*}}
\newcommand{\Vstar}    {V^{*}}
\newcommand{\Xstar}    {X^{*}}

\newcommand{\eperp}     {e^{\bot}}
\newcommand{\xboldstar}{ {\mbox{\bf x}}^{*}}
\newcommand{\yboldstar}{ {\mbox{\bf y}}^{*}}
\newcommand{\ystarhat}{ \hat{y}^{*}}
\newcommand{\yboldstarhat}{ \hat{\mbox{\bf y}}^{*}}

\newcommand{\betastar} {\beta^{*}}
\newcommand{\etastar}  {\eta^{*}}

\newcommand{\Pbarstar} {\bar{P}^{*}}

\newcommand{\Aplus}    {A^{+}}

\newcommand{\Amins}    {A^{-}}
\newcommand{\Xmins}    {X^{-}}
\newcommand{\Zmins}    {Z^{-}}

\newcommand{\Zinv}     {Z^{-1}}
\newcommand{\Zginv}    {Z^{-}}


\newcommand{\Zboldinv} {\mbox{\bf Z}^{-1}}

\newcommand{\Qboldginv}{\mbox{\bf Q}^{-}}
\newcommand{\Rboldginv}{\mbox{\bf R}^{-}}
\newcommand{\Zboldginv}{\mbox{\bf Z}^{-}}

\newcommand{\rank}     {\mbox{rank}}
\newcommand{\trace}    {\mbox{trace}}

\newcommand{\smint}{\! \int \!}
\newcommand{\Gambt}{\tilde{\mbox{\boldmath${\Gamma}$}}}

\newcommand{\cvu}{a}
\newcommand{\cvv}{b}
\newcommand{\cfu}{\xi}
\newcommand{\cfv}{\eta}
\newcommand{\Vboldt}{\tilde{\Vbold}}
\newcommand{\funcov}{v}
\newcommand{\funcovh}{\hat{v}}
\newcommand{\Funcov}{V}
\newcommand{\Funcovh}{\hat{V}}
\newcommand{\cancov}{{\mbox{\tt ccorsq}}^{pop}}
\newcommand{\cancovh}{\mbox{\tt ccorsq}}

\newcommand{\Splus}{S-PLUS}
\newcommand{\df}{\mbox{\tt df}}
\newcommand{\etal}{et al.\ }

\setbeamercovered{dynamic}

\newcommand{\Lang}[1]{\operatorname{\text{\textsc{#1}}}}

\newcommand{\Class}[1]{\operatorname{\mathchoice
  {\text{\sf \small #1}}
  {\text{\sf \small #1}}
  {\text{\sf #1}}
  {\text{\sf #1}}}}

\newcommand{\NumSAT}      {\text{\small\#SAT}}
\newcommand{\NumA}        {\#_{\!A}}

\newcommand{\barA}        {\,\bar{\!A}}

\newcommand{\Nat}{\mathbb{N}}
\newcommand{\Set}[1]{\{#1\}}

\pgfdeclaremask{tu}{beamer-tu-logo-mask} \pgfdeclaremask{computer}{beamer-computer-mask}
\pgfdeclareimage[interpolate=true,mask=computer,height=2cm]{computerimage}{beamer-computer}
\pgfdeclareimage[interpolate=true,mask=computer,height=2cm]{computerworkingimage}{beamer-computerred}
\pgfdeclareimage[mask=tu,height=.5cm]{logo}{beamer-tu-logo}

%\logo{\pgfuseimage{logo}}

\colorlet{redshaded}{red!25!bg} \colorlet{shaded}{black!25!bg} \colorlet{shadedshaded}{black!10!bg}
\colorlet{blackshaded}{black!40!bg}

\colorlet{darkred}{red!80!black} \colorlet{darkblue}{blue!80!black}
\colorlet{darkgreen}{green!80!black}

\def\radius{0.96cm}
\def\innerradius{0.85cm}

\def\softness{0.4}
\definecolor{softred}{rgb}{1,\softness,\softness}
\definecolor{softgreen}{rgb}{\softness,1,\softness}
\definecolor{softblue}{rgb}{\softness,\softness,1}

\definecolor{softrg}{rgb}{1,1,\softness}
\definecolor{softrb}{rgb}{1,\softness,1}
\definecolor{softgb}{rgb}{\softness,1,1}

\newcommand{\Bandshaded}[2]{
  \color{shadedshaded}
  \pgfmoveto{\pgfxy(-0.5,0)}
  \pgflineto{\pgfxy(-0.6,0.1)}
  \pgflineto{\pgfxy(-0.4,0.2)}
  \pgflineto{\pgfxy(-0.6,0.3)}
  \pgflineto{\pgfxy(-0.4,0.4)}
  \pgflineto{\pgfxy(-0.5,0.5)}
  \pgflineto{\pgfxy(4,0.5)}
  \pgflineto{\pgfxy(4.1,0.4)}
  \pgflineto{\pgfxy(3.9,0.3)}
  \pgflineto{\pgfxy(4.1,0.2)}
  \pgflineto{\pgfxy(3.9,0.1)}
  \pgflineto{\pgfxy(4,0)}
  \pgfclosepath
  \pgffill

  \color{black}
  \pgfputat{\pgfxy(0,0.7)}{\pgfbox[left,base]{#1}}
  \pgfputat{\pgfxy(0,-0.1)}{\pgfbox[left,top]{#2}}
}

\newcommand{\Band}[2]{
  \color{shaded}
  \pgfmoveto{\pgfxy(-0.5,0)}
  \pgflineto{\pgfxy(-0.6,0.1)}
  \pgflineto{\pgfxy(-0.4,0.2)}
  \pgflineto{\pgfxy(-0.6,0.3)}
  \pgflineto{\pgfxy(-0.4,0.4)}
  \pgflineto{\pgfxy(-0.5,0.5)}
  \pgflineto{\pgfxy(4,0.5)}
  \pgflineto{\pgfxy(4.1,0.4)}
  \pgflineto{\pgfxy(3.9,0.3)}
  \pgflineto{\pgfxy(4.1,0.2)}
  \pgflineto{\pgfxy(3.9,0.1)}
  \pgflineto{\pgfxy(4,0)}
  \pgfclosepath
  \pgffill

  \color{black}
  \pgfputat{\pgfxy(0,0.7)}{\pgfbox[left,base]{#1}}
  \pgfputat{\pgfxy(0,-0.1)}{\pgfbox[left,top]{#2}}
}

\newcommand{\BaenderNormal}
{%
  \pgfsetlinewidth{0.4pt}
  \color{black}
  \pgfputat{\pgfxy(0,5)}{\Band{input tapes}{}}
  \pgfputat{\pgfxy(0.35,4.6)}{\pgfbox[center,base]{$\vdots$}}
  \pgfputat{\pgfxy(0,4)}{\Band{}{}}

  \pgfxyline(0,5)(0,5.5)
  \pgfxyline(1.2,5)(1.2,5.5)
  \pgfputat{\pgfxy(0.25,5.25)}{\pgfbox[left,center]{$w_1$}}

  \pgfxyline(0,4)(0,4.5)
  \pgfxyline(1.8,4)(1.8,4.5)
  \pgfputat{\pgfxy(0.25,4.25)}{\pgfbox[left,center]{$w_n$}}
  \ignorespaces}

\newcommand{\BaenderZweiNormal}
{%
  \pgfsetlinewidth{0.4pt}
  \color{black}
  \pgfputat{\pgfxy(0,5)}{\Band{Zwei Eingabebänder}{}}
  \pgfputat{\pgfxy(0,4.25)}{\Band{}{}}

  \pgfxyline(0,5)(0,5.5)
  \pgfxyline(1.2,5)(1.2,5.5)
  \pgfputat{\pgfxy(0.25,5.25)}{\pgfbox[left,center]{$u$}}

  \pgfxyline(0,4.25)(0,4.75)
  \pgfxyline(1.8,4.25)(1.8,4.75)
  \pgfputat{\pgfxy(0.25,4.5)}{\pgfbox[left,center]{$v$}}
  \ignorespaces}

\newcommand{\BaenderHell}
{%
  \pgfsetlinewidth{0.4pt}
  \color{black}
  \pgfputat{\pgfxy(0,5)}{\Bandshaded{input tapes}{}}
  \color{shaded}
  \pgfputat{\pgfxy(0.35,4.6)}{\pgfbox[center,base]{$\vdots$}}
  \pgfputat{\pgfxy(0,4)}{\Bandshaded{}{}}

  \color{blackshaded}
  \pgfxyline(0,5)(0,5.5)
  \pgfxyline(1.2,5)(1.2,5.5)
  \pgfputat{\pgfxy(0.25,5.25)}{\pgfbox[left,center]{$w_1$}}

  \pgfxyline(0,4)(0,4.5)
  \pgfxyline(1.8,4)(1.8,4.5)
  \pgfputat{\pgfxy(0.25,4.25)}{\pgfbox[left,center]{$w_n$}}
  \ignorespaces}

\newcommand{\BaenderZweiHell}
{%
  \pgfsetlinewidth{0.4pt}
  \color{black}
  \pgfputat{\pgfxy(0,5)}{\Bandshaded{Zwei Eingabebänder}{}}%
  \color{blackshaded}
  \pgfputat{\pgfxy(0,4.25)}{\Bandshaded{}{}}
  \pgfputat{\pgfxy(0.25,4.5)}{\pgfbox[left,center]{$v$}}
  \pgfputat{\pgfxy(0.25,5.25)}{\pgfbox[left,center]{$u$}}%

  \pgfxyline(0,5)(0,5.5)
  \pgfxyline(1.2,5)(1.2,5.5)

  \pgfxyline(0,4.25)(0,4.75)
  \pgfxyline(1.8,4.25)(1.8,4.75)
  \ignorespaces}

\newcommand{\Slot}[1]{%
  \begin{pgftranslate}{\pgfpoint{#1}{0pt}}%
    \pgfsetlinewidth{0.6pt}%
    \color{structure}%
    \pgfmoveto{\pgfxy(-0.1,5.5)}%
    \pgfbezier{\pgfxy(-0.1,5.55)}{\pgfxy(-0.05,5.6)}{\pgfxy(0,5.6)}%
    \pgfbezier{\pgfxy(0.05,5.6)}{\pgfxy(0.1,5.55)}{\pgfxy(0.1,5.5)}%
    \pgflineto{\pgfxy(0.1,4.0)}%
    \pgfbezier{\pgfxy(0.1,3.95)}{\pgfxy(0.05,3.9)}{\pgfxy(0,3.9)}%
    \pgfbezier{\pgfxy(-0.05,3.9)}{\pgfxy(-0.1,3.95)}{\pgfxy(-0.1,4.0)}%
    \pgfclosepath%
    \pgfstroke%
  \end{pgftranslate}\ignorespaces}

\newcommand{\SlotZwei}[1]{%
  \begin{pgftranslate}{\pgfpoint{#1}{0pt}}%
    \pgfsetlinewidth{0.6pt}%
    \color{structure}%
    \pgfmoveto{\pgfxy(-0.1,5.5)}%
    \pgfbezier{\pgfxy(-0.1,5.55)}{\pgfxy(-0.05,5.6)}{\pgfxy(0,5.6)}%
    \pgfbezier{\pgfxy(0.05,5.6)}{\pgfxy(0.1,5.55)}{\pgfxy(0.1,5.5)}%
    \pgflineto{\pgfxy(0.1,4.25)}%
    \pgfbezier{\pgfxy(0.1,4.25)}{\pgfxy(0.05,4.15)}{\pgfxy(0,4.15)}%
    \pgfbezier{\pgfxy(-0.05,4.15)}{\pgfxy(-0.1,4.2)}{\pgfxy(-0.1,4.25)}%
    \pgfclosepath%
    \pgfstroke%
  \end{pgftranslate}\ignorespaces}

\newcommand{\ClipSlot}[1]{%
  \pgfrect[clip]{\pgfrelative{\pgfxy(-0.1,0)}{\pgfpoint{#1}{4cm}}}{\pgfxy(0.2,1.5)}\ignorespaces}

\newcommand{\ClipSlotZwei}[1]{%
  \pgfrect[clip]{\pgfrelative{\pgfxy(-0.1,0)}{\pgfpoint{#1}{4.25cm}}}{\pgfxy(0.2,1.25)}\ignorespaces}


\AtBeginSection[]{\frame{\frametitle{Outline}\tableofcontents[current]}}


\title{Multivariate and Functional Principal Components without Eigenanalysis}

\author{Jim Ramsay, McGill University \\
        Workshop on New Perspectives in FDA \\
        Caserta, Italy \\
        26 September 2012}
\date{}

\begin{document}

\begin{frame}

\maketitle

%\vspace{-1.9cm}
%\begin{center}
%\includegraphics[width=2.5in]{figs/maternSig51.png}
%\end{center}

\end{frame}

%  ---------------------------------------------------------------------
%  ---------------------------------------------------------------------

\section[Overview]{Introduction}

%  --------------------------------------------------------------------

\begin{frame}

\frametitle{Where we are going}

\bi
  \item Principal components analysis has some serious limitations
  \item Principal components are structural parameters; principal component scores are nuisance parameters
  \item Parameter cascading defines nuisance parameters as smooth functions of structural parameters,
  and then optimizes fit with respect to only structural parameters
  \item What this means for multivariate PCA
  \item Functional PCA
  \item Functional PCA with registration
\ei

\end{frame}

%  --------------------------------------------------------------------

\begin{frame}

\frametitle{PCA: The essential idea}

\bi
  \item We have a $N$ by $n$ data matrix $\Ybold$.
  \item We propose the reduced rank bilinear model
  \[
    \Ybold = \Fbold \Abold
  \]
  where
    \bi
      \item $\Abold$ is a $K$ by $n$ matrix of principal component coefficients, with $K << n$
      \item $\Fbold$ is a $N$ by $K$ matrix of principal component scores
    \ei
  \item Usually $N >> n$, and the factor scores are interesting, but it's $\Abold$ that tells
  us what the core $K$ components of variation are, to within a full rank linear transformation.
\ei

\end{frame}

%  --------------------------------------------------------------------

\begin{frame}

\frametitle{Estimation of the PCA model}

\bi
  \item We can use the singular value decomposition to get both $\Abold$ and $\Fbold$ in one step,
  \item but the usual procedure is to extract $\Abold$ from the eigenanalysis of $N^{-1} \Ybold' \Ybold$
  or the corresponding correlation matrix $\Rbold$
  \item and then use regression analysis to obtain
  \[
    \Fbold = \Ybold \Abold' (\Abold' \Abold)^{-1}
  \]
\ei

\end{frame}

%  --------------------------------------------------------------------

\begin{frame}

\frametitle{What are nuisance parameters?}

\bi
  \item Many models contain parameters that are not themselves of much direct interest, but which must be in the model because they explain a substantial amount of the variance. They may also have these characteristics:
  \bi
    \item Their number may be far greater than that of the structural parameters, which are of direct interest.
    \item Their number may also vary in proportion to the amount of data.
    \item Each of them typically controls the fit to only a small numbers of observations.
  \ei
  \item Principal component scores are an example, as are ability parameters in a test theory model.
  \item Often nuisance parameters are treated as random effects, as in the hierarchical linear model.
\ei
\end{frame}

%  ---------------------------------------------------------------------
%  ---------------------------------------------------------------------

\section[PCA Limitations]{What's not to love about PCA?}

%  --------------------------------------------------------------------

\begin{frame}

\frametitle{The least squares issue}

\bi
  \item PCA defined this way is a least squares fit to the data, even when the variables may be
  \bi
    \item discrete or even binary
    \item bounded between two limits
    \item otherwise severely non-normal
    \item have outliers or be heavy-tailed
  \ei

\ei

\end{frame}

%  --------------------------------------------------------------------

\begin{frame}

\frametitle{The eigenvector issue}

\bi
  \item The rows of $\Abold$ are proportional to the eigenvectors of $\Rbold$, and
  span the subspace of dimension $K$.
  \item Social science statisticians all know that they are seldom of of great interest,
  and that they can be either rotated or linearly transformed to define new vectors
  spanning the same space that are much more easily interpreted or labelled
  \item But generations of statisticians elsewhere learn about PCA from well-known textbooks that
  never mention rotation, and attempt to extract meaning from eigenvectors.
  \item Eigenvectors are just biproducts of the the computation process, and therefore of
  computational interest only.
  \item Eigenvalues are only informative in terms of the cumulative sums.
  \item PCA is widely misunderstood as a consequence.
\ei

\end{frame}

%  --------------------------------------------------------------------

\begin{frame}

\frametitle{The estimation issue}

\bi
  \item PCA as we know it treats the estimation of $\Abold$ and $\Fbold$ completely symmetrically.
  \item But PC scores are effectively nuisance parameters or random effects, while PC coefficients are structural parameters or fixed effects.
  \item We know that the need to estimate large numbers of nuisance or random parameters can seriously degrade the quality of estimates of structural or fixed parameters.
  \item Test theory, hierarchical linear modeling and many other methods have evolved to control
  this impact so as to improve the estimation core fixed-size parameters like $\Abold.$
  \item The usual approach is to apply some smoothing or regularization to the high-dimensional nuisance parameters.  How can we do this in PCA?
\ei

\end{frame}

%  ---------------------------------------------------------------------
%  ---------------------------------------------------------------------

\section{Parameter Cascading}

%  --------------------------------------------------------------------

\begin{frame}

\frametitle{How does it work?}

\bi
  \item Parameter cascading defines nuisance parameters as \emph{smooth} functions of
  structural parameters.
  \item The PCA regression equation
  \[
    \Fbold(\Abold) = \Ybold \Abold' (\Abold' \Abold)^{-1}
  \]
  already defines PC scores as functions of PC coefficients.
  \item We add smoothness by attaching a penalty term such as
  \[
    \Fbold(\Abold) = \Ybold \Abold' (\Abold' \Abold)^{-1} + \lambda \trace [ \Fbold' \Pbold \Fbold ]
  \]
  \item Order $K$ matrix $\Pbold$ can be a projection onto the complement of some pre-defined subspace or special pattern.
  \item The roughness penalty measure the extent to which the rows of $\Fbold$ fail to satisfy some
  constraint.
  \item Smoothing parameter $\lambda \geq 0$ allows us to control the emphasis that we place on
  the PC scores having this particular structure.
\ei

\end{frame}

%  --------------------------------------------------------------------

\begin{frame}

\frametitle{The fitting criterion for $\Abold$}

\bi
  \item This is defined in terms of only the PC coefficients $\Abold$.
  \item For example, we could use
  \[
    G(\Abold) = -\sum_j^n \ln L_j(\Abold|\ybold_j)
  \]
  where $-\ln L_j$ is the negative log likelihood appropriate to variable $j$ and
  defined by data $N$-vector $\ybold_j$.
  \item Note that the gradient of $G$ will depend on $\Abold$ both directly through its the partial derivative, and also via the $N$ functions $\fbold_i(\Abold)$.
  \item That is, we need the total derivative
  \[
    \frac{dG}{d \Abold} = \frac{\partial G}{\partial \Abold} + \sum_i^N \frac{dF_i}{d \Abold}
  \]
  where the fitting functions $F_i$ are specific to the data $n$-vectors $\ybold^i$.
\ei

\end{frame}

%  --------------------------------------------------------------------

\begin{frame}

\frametitle{Bringing in the Implicit Function Theorem}

\bi
  \item These computations seem possible when we have an explicit formula for how
  the PC scores $\fbold_i$ depend on $\Abold$, as we have just proposed.
  \item But what about when the functional relationship is defined by the numerical
  optimization of fitting functions $F_i$ with respect to $\fbold_i$?  How do we get
  $\frac{dF_i}{d \Abold}$ in this case?
  \item In this case we bring in the Implicit Functional Theorem, a result that
  should be much better known than it is.
  \item It states that the total derivative of $G$ is
  \[
    \frac{dG}{d \Abold} = \frac{\partial G}{\partial \Abold} -
    \sum_i^N (\frac{\partial G}{\partial \fbold_i}) (\frac{\partial^2 F_i}{\partial^2 \fbold_i})^{-1}
    (\frac{\partial^2 F_i}{\partial \fbold_i \partial \Abold})
  \]
\ei

\end{frame}

\end{document}
